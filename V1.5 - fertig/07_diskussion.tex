\newpage
\section{Diskussion}
\label{sec:diskussion}


\subsection{Bedeutung der Driftkorrektur}
Der Drift wurde als linearer Zusammenhang vom Messgerät durch Kalibrierung erfasst. Die Einbeziehung des Massen- und Volumendrifts erlaubt es die wiederkehrende Abweichung des Messgerätes zu berücksichtigen und damit eine größere Genauigkeit des erhaltenen Ergebnisses. Eine absolute Abweichung wird dabei durch Multiplikation mit der Messdauer erhalten.



\subsection{Bewertung der Feuchtigkeit}

\textbf{Isopropanol/2-Propanol}\\
Die Feuchtigkeitsmesswerte des Isopropanols sind in der Tabelle \ref{tab:Messwerte2Propanol} einzusehen. Der mittlere gemessene Wassergehalt von 0,0407\% ist sehr gering. Isopropanol ist in jedem Verhältnis mit Wasser mischbar und somit ist das Lösen von Luftfeuchtigkeit nicht auszuschließen.\cite{isopropanol} Zudem ist Isopropanol auch im Handel mit $99,95\%$ Reinheit erhältlich, was die gemessenen $0,04\%$ Wassergehalt bestätigen könnte.\cite{isoprop} Daher ist der Messwert plausibel.\\

\textbf{Polyamid}\\
Die Feuchtigkeitsmesswerte des Polyamids sind in der Tabelle \ref{tab:MesswertePoliamid} einzusehen. Der mittlere gemessene Wassergehalt von 2,1203\% ist deutlich höher als beim Isopropanol. Dieser Unterschied basiert auf der Zusammensetzung des Polyamid und ist abhängig von dessen Konzentration an Amidgruppen im kristallinen Gefüge. Somit kommt des dazu, dass bei Umgebungsluft Polyamid je nach genauem Gefüge und je nach betrachteter Quelle, beispielsweise bei PA 6, $2,5-3,5\%$ Wassergehalt möglich sind.\cite{Wikipedia.2020,Kaiser.2006} Als Gegenbeispiel besitzt PA 12 bei Umgebungsluft einen Wassergehalt von $0,2-0,5\%$.\cite{Wikipedia.2020} Je nachdem welche Art von Polyamid-Probe untersucht wurde erscheint der Messwert als mehr oder weniger plausibel.\\

\textbf{Risiken von Wasser bei der Polymerverarbeitung}\\
Die gebräuchlichsten Alltagspolymere sind Thermoplaste. Sie werden zur Verarbeitung und Umformung meist erwärmt. Wie die Polyamid-Probe zeigt können Kunststoffe schon bei Umgebungsluft in geringen Mengen aufnehmen. Das hat zur Folge, dass ähnlich wie beim Werkstoff Holz, der Kunststoff aufquellen kann. Eine mögliche Folge ist, dass sich Bauteile einer Volumenzunahme von 0,3 \% pro 1 \% Wasseraufnahme zur Folge haben können.\cite{Kaiser.2006} Je nach Einsatzgebiet könne  hierbei Stabilität und Sicherheit gefährdet sein. Im selben Zusammenhang können sich durch das Quellen oder Schwinden des Kunststoffes Einschlüsse bilden, welche ebenfalls den Kunststoff in seiner Struktur verändern können. Betrachtet man Wasser nicht nur als Quell- sondern auch als Lösungsmittel so ist durchaus auch eine Extraktion von Inhaltsstoffen, wie Additive möglich, welche den Kunststoff nachhaltig verändern können. Auch die Veränderung von weiteren physikalischen Eigenschaften ist dabei möglich, wie zum Beispiel die Veränderung der Leitfähigkeit.\cite{polymermerse} 