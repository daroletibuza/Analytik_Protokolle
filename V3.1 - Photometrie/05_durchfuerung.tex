\section{Durchführung}
\label{sec:durchfuerung}
Als erstes muss das Spektralphotometer eingeschaltet werden, da eine \glqq Vorwärmzeit\grqq\, von ca. \SI{15}{\minute} empfohlen wird. Diese erlaubt die Ausbildung eines thermischen Gleichgewichts. Die Kalibrierung geschieht mit einer Leerprobe. Diese soll die Matrix der im Anschluss betrachteten Lösungen möglichst genau nachbilden. Daher werden \SI{20}{\milli\liter} Reinstwasser mit jeweils \SI{200}{\micro\liter} Kaliumnatriumtartratlösung und \SI{200}{\micro\liter} NEßLER-Reagenz vermischt. Nun kann aus der Ammonium-Stammlösung die Ammonium-Standardlösung durch Verdünnen hergestellt werden. Dazu wird ein Milliliter der Stammlösung in einem Maßkolben mit Reinstwasser auf \SI{100}{\milli\liter} aufgefüllt. Aus der Standardlösung werden im Anschluss die Kalibrierlösungen durch nochmaliges Verdünnen Hergestellt. Ziel sind dabei jeweils drei Kalibrierlösungen einheitlicher Beschaffenheit. Zur Kalibrierung dienen die Konzentrationen von \SI{0,5}{\micro\gram\per\milli\liter}, \SI{1,5}{\micro\gram\per\milli\liter} und \SI{3,0}{\micro\gram\per\milli\liter} und einem Volumen von \SI{20}{\milli\liter}. Den Kalibrierlösungen werden, wie den Analyseproben auch, jeweils \SI{200}{\micro\liter} Kaliumnatriumtartratlösung und \SI{200}{\micro\liter} NEßLER-Reagenz hinzugefügt. Daraufhin sollten sich die Lösungen gleicher Konzentration auch im gleichen gelben Farbton färben. Die Analyseproben werden ganz analog vorbereitet. Vor der Messung sollten die Lösungen mindestens \SI{15}{\minute} ruhen, um eine vollständige Komplexierung der Ammoniumionen zu ermöglichen. Die Analyse der Kalibrierlösungen und Proben erfolgt durch Befüllen und anschließendes Einsetzen der Küvetten in den Probenraum des Spektralphotometers. Die Absorbanzmesswerte werden am Display abgelesen und notiert. 
Sollte aus Genauigkeitsgründen eine einzige Küvette für alle Messungen genutzt werden, so ist diese nach vor jeder neuen Füllung mit der neuen Lösung zu spülen, um Kontaminationen durch vorhergehende Proben zu vermeiden.