\section{Durchführung}
\label{sec:durchfuerung}
Tatsächlich wurde das Praktikum nicht durch die Autoren dieses Protokolls durchgeführt. Aus diesem Grund wird an dieser Stelle auf die Beschreibung der Versuchsdurchführung nur in vereinfachter Form dargestellt.\\
Zunächst würde die Inbetriebnahme des Titrators gestartet werden und es wird auf die Titratorverbindung zum Rechner gewartet. Parallel dazu würde nun der Spülgasstrom mit Stickstoff eingeschaltet werden und der Ofen für den Versuchsteil 2 auf \SI{300}{\celsius} vorgeheizt werden. \\
Sind die Vorbereitungen getroffen würde infolgedessen die Kalibrierung der KF-Lösung mit \SI{5}{\micro \liter} einer Wasserstandardsubstanz erfolgen. Über die ermittelten Messwerte kann somit der Titer der Titrierlösung bestimmt werden.
Als nächstes würden nun zwei Flüssigproben nach ähnlichem Vorgehen auf ihren Wassergehalt untersucht werden.\\
Im zweiten Versuchsteil würde nun eine Feststoffprobe über die Ofentechnik der Wassergehalt bestimmt. Der Ofen würde hierfür nun auf \SI{200}{\celsius} abgesenkt werden. Danach wird die entsprechende Feststoffprobe in den Ofen gelegt und die Titration gestartet.\\
Im Anschluss dessen würden die entsprechenden Datensätze am PC mit \textsc{LabX} ausgewählt und gedruckt werden.