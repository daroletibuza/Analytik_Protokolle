\section{Theorie}
\label{sec:theorie}

\subsection{Lambert-Beer'sches Gesetz}
Als Messverfahren für die Ammonium-Bestimmung wird in diesem Versuch die Photometrie genutzt. Photometrie beschreibt dabei die Messung der Absorbanz einer bestimmten Wellenlänge. \linebreak
Grundlage für den Zusammenhang von Absorbanz und Konzentration stellt hierbei das \textsc{Lambert-Beer}'sche Gesetz dar (siehe Gl.\eqref{gl:3}). Aus diesem ist ersichtlich, dass sich Absorbanz $A \, [-]$ und Konzentration $c \, \left[\si{\mol\per\liter}\right]$ bei konstanter Schichtdicke $d \, \left[\si{\centi \meter}\right]$ und  konstantem Absorptionskoeffizient $\alpha \, \left[\si{\liter \per \mol \per \centi \meter}\right]$ proportional verhalten.\\
Die Anwendung des \textsc{Lambert-Beer}'sche Gesetzes setzt dabei monochromatisches Lichtstrahlung voraus, sowie die Messung einer idealen Lösung.
	\begin{flalign}
	\label{gl:3}
		A	&= \alpha(\lambda)*c*d\\
		A	&= pT = -\log(T)
	\end{flalign}
	
\pagebreak

Die Absorbanz selbst beschreibt den negativ dekadischen Logarithmus der Durchlässigkeit $T$ eines Mediums und stellt somit ein Maß für die absorbierte elektromagnetische Strahlung dar. Die Durchlässigkeit $T$ definiert sich als Reststrahlung in Form der Intensität $I$ im Verhältnis zur Ausgangsintensität $I_0$ (siehe Gl.\eqref{gl:4}).
\begin{flalign}
\label{gl:4}
	T &= \frac{I}{I_0}
\end{flalign}

\subsection{Entwicklung photometrischer Methoden}
Um bestimmte Atome/Moleküle/Ionen in Lösungen mittels Photometrie quantifizieren zu können, ist die Bestimmung der charakteristischen Wellenlänge nötig. Um diese identifizieren zu können, werden spektroskopische Messungen des gesuchten Atoms/Moleküls/Ions durchgeführt. Das bei einer bestimmten Wellenlänge auftretende Absorptionsmaximum entspricht, dann der sensitiven Wellenlänge für die photometrische Untersuchung. Da die Lichtabsorption, nach dem \textsc{Lambert-Beer}'schen Gesetz, von der Konzentration abhängig ist, kann diese Konzentration einer Probe mittels Kalibriergerade bestimmt werden.

\subsection{Voraussetzungen für die Absorption im UV/-VIS-Bereich}
Um überhaupt eine Absorption durch ein Atom/Molekül/Ion garantieren zu können, ist das Vorhandensein von Valenzelektronen von Bedeutung. Die UV/VIS-Spektroskopie basiert auf Elektronenübergängen nach Zuführung von elektromagnetischer Energie. Diese Energie benötigt einen bestimmten Betrag, um die Valenzelektronen in einen angeregten Zustand zu versetzen. Dieser Betrag, um ein Elektron aus den energieärmeren Zustand HOMO in den energiereicheren Zustand des LUMO zu versetzen, ist für jedes Atom/Molekül/Ion charakteristisch. \\
Die Strahlung, in Form von elektromagnetischer Energie, liegt bei diesem Verfahren im ultravioletten bis sichtbarem Bereich.

\subsection{Absorption mit Ammonium-Ionen}
Ammonium-Ionen erscheinen in Lösung selbst nicht farbig. Grund hierfür ist, dass diese Ionen Licht nicht im sichtbaren Spektrum absorbieren, sondern in einem sehr kurzwelligen und energiereichen Bereich. Dieser kurzwellige Bereich wird in diesem Versuch vom Photometer nicht detektiert. Das hat zur Folge, dass die Ammonium-Ionen mittels NEßLER-Reagenz zu einem rot-braunen Komplex in der Lösung koordinieren. Dieser ist mit einer Wellenlänge von \SI{425}{\nano \meter} für das Messgerät erfassbar.

