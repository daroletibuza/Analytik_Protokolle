\newpage

\section{Ergebnisse und Berechnungen}
\label{sec:ergebnisse}

\begin{figure}[h!]
	\begin{center}
		\resizebox{0.9\textwidth}{!}{
			\begin{tikzpicture}[trim axis left, trim axis right]
			\begin{axis}[
			axis lines = left,
			width = 15cm,
			height = 11cm,
			xmin = 0,
			xmax = 3,
			%	ymin = -0.1,
			%	ymax = 0,
			%	ytick = {-4.5,-4,...,-1},
		%	xtick = {-10,-9,...,20},
			ylabel={Absorbanz},
			%y label style={at={(0.25,0.40)}, rotate=90},
			xlabel={Ammonium-Konzentration \si{\milli \gram \per \milli \liter}},
			legend style={at={(0.75,0.45)},anchor=west},
		%	y dir = reverse,
			]
			\addplot +[only marks,mark options={draw=black,fill=black}] table {data/kalibriergerade.dat};
			\addplot +[mark=none, dashed, black, domain=0.5:3] {0.106175439*x-0.034070175	
			};
			
			\legend{Kalibrierpunkte, Kalibriergerade};
			\end{axis}
			\end{tikzpicture}
		}
			\caption{Absorbanz in Abhängigkeit der Ammonium-Konzentration}
		\label{dia:kalibrierung}
	\end{center}
\end{figure}
\FloatBarrier
\vspace*{-5mm}

\textbf{Berechnung der Ammonium-Konzentration aus der Kalibriergeraden}
\begin{flalign}
		A 			&= f(c_{\ce{NH4+}}) = 0,106*c_{\ce{NH4+}}-0,034\\[1mm]
	c_{\ce{NH4+}} 	&= \frac{A+0,034}{0,106} \,\si{\milli \gram \per \liter}\\
					&= \frac{0,029+0,034}{0,106} \,\si{\milli \gram \per \liter}\\
					&= \underline{\SI{0,59}{\milli \gram \per \liter}}
\end{flalign}


\textbf{Berechnung der Konzentration von Stickstoff aus der Ammonium-Konzentration}
\begin{flalign}
 c_{\ce{N}}	 	&= n\%_{\ce{N}} *c_{\ce{NH4+}}\\
 				&= \frac{M_{\ce{N}}}{_{\ce{NH4+}}}*c_{\ce{NH4+}}\\
 				&= \frac{\SI{14,00067}{\gram \per \mol}}{\SI{18,03846}{\gram \per \mol}}*\SI{0,69}{\milli \gram \per \liter}\\
 				&= 0,78*\SI{0,69}{\milli \gram \per \liter}\\
 				&=\underline{ \SI{0,54}{\milli \gram \per \liter}}
\end{flalign}

\textbf{Berechnung des Mittelwertes:}
\begin{flalign}
\label{Gl:Mittelwert-Beispielrechnung1}
\bar{x} &= \frac{\sum_{n=1}^{N}x_n}{N}\\ 
%\label{Gl:Mittelwert-Beispielrechnung2}
\bar{x} &= \frac{0,02+0,017+0,021}{3}\\
&= \underline{0,019}
\end{flalign}

\textbf{Berechnung der Standardabweichung:}
\begin{flalign}\label{Gl:Standardabweichung-Beispielrechnung}
s &= \sqrt{\frac{\sum_{n=1}^{N}(x_n-\bar{x})^2}{N-1}}
\\
&= \sqrt{\frac{(0,59\si{\milli\gram\per\liter}-0,69\si{\milli\gram\per\liter})^2+(0,91\si{\milli\gram\per\liter}-0,69\si{\milli\gram\per\liter})^2+(0,58\si{\milli\gram\per\liter}-0,69\si{\milli\gram\per\liter})^2}{2}}\\
&= \underline{0,19\si{\milli\gram\per\liter}}
\end{flalign}
%	\end{footnotesize}

\textbf{Berechnung der relativen Standardabweichung:}
\begin{flalign}\label{gl:S_rel}
	s_{rel}&=\frac{s}{\bar{x}}\\
	&=\frac{\SI{0,19}{\milli\gram\per\liter}}{\SI{0,69}{\milli\gram\per\liter}}\\
	&=\underline{28\%}
\end{flalign}

\textbf{Berechnung des Vertrauensintervalls:}
\begin{flalign}
conf(\bar{x}) 	&= \bar{x}\pm \frac{t}{\sqrt{N}}*s				\\
conf(\bar{x})	&= \SI{0,69}{\milli \gram \per \liter}\pm \frac{4,303}{\sqrt{3}}*\SI{0,19}{\milli \gram \per \liter}\\
&= \underline{\SI{0,69}{\milli \gram \per \liter}\pm \SI{0,47}{\milli \gram \per \liter}}
\end{flalign}

\textbf{Grenzwerttest für Trinkwasser (\SI{0,5}{\milli \gram \per \liter} \ce{NH4+}):}\label{sec:GWtest}
\begin{flalign}
t_{\text{EMP},95}&=\frac{\left|\bar{x}-x_{Grenz}\right|}{s}*\sqrt{N}\\[2mm]
&=\frac{\SI{0,69}{\milli\gram\per\liter}-\SI{0,50}{\milli\gram\per\liter}}{\SI{0,47}{\milli\gram\per\liter}}*\sqrt{3}\\
&= \underline{1,73}
\end{flalign}
%\newpage
Der tabellierte Wert für $t_{\text{CRIT}}$ für einen einseitigen Test mit einer Sicherheit von 95\% bei einem Freiheitsgrad von 2 lautet 2,920.\\
Um mit 95\% Sicherheit sagen zu können, dass der Messwert unterhalb des Grenzwertes liegt gilt folgendes Kriterium:
\vspace{-5mm}
\begin{flalign}
	t_{\text{EMP}} &< -t_{\text{CRIT}} \\
	1,73			&< -2,92 \tag{falsche Aussage}
\end{flalign}\vspace{-2mm} 
Somit liegt der Messwert laut statistischer Auswertung nicht unterhalb der einzuhaltenden Höchstkonzentration von Trinkwasser.
