\newpage

\vspace*{10mm}

\textbf{\textcolor{red}{Aufgaben:}}
\textcolor{red}{
	\begin{itemize}
		\item Verdünnungsrechnung
		\item Aufpassen bei Konzentrationsangaben wegen Runden
	\end{itemize}
}

\section{Ergebnisse und Berechnungen}
\label{sec:ergebnisse}


\textbf{Berechnung des Mittelwertes:}
\begin{flalign}
\label{Gl:Mittelwert-Beispielrechnung1}
\bar{x} &= \frac{\sum_{n=1}^{N}x_n}{N}\\ 
%\label{Gl:Mittelwert-Beispielrechnung2}
\bar{x} &= \frac{0,02+0,017+0,021}{3}\\
&= \underline{0,019}
\end{flalign}

\textbf{Berechnung der Standardabweichung:}
\begin{flalign}\label{Gl:Standardabweichung-Beispielrechnung}
s &= \sqrt{\frac{\sum_{n=1}^{N}(x_n-\bar{x})^2}{N-1}}
\end{flalign}
%	\begin{footnotesize}
\begin{flalign}
s &= \sqrt{\frac{(0,594\si{\milli\gram\per\liter}-0,694\si{\milli\gram\per\liter})^2+(0,914\si{\milli\gram\per\liter}-0,694\si{\milli\gram\per\liter})^2+(0,575\si{\milli\gram\per\liter}-0,694\si{\milli\gram\per\liter})^2}{2}}\\
&= \underline{0,19\si{\milli\gram\per\liter}}
\end{flalign}
%	\end{footnotesize}

\textbf{Berechnung der relativen Standardabweichung:}

\begin{flalign}\label{gl:S_rel}
	s_{rel}&=\frac{s}{\bar{x}}\\
	&=\frac{\SI{0,19}{\milli\gram\per\liter}}{\SI{0,69}{\milli\gram\per\liter}}\\
	&=\underline{28\%}
\end{flalign}

\textbf{Berechnung des Vertrauensintervalls:}\\
\begin{flalign}
conf(\bar{x}) 	&= \bar{x}\pm \frac{t}{\sqrt{N}}*s				
\end{flalign}
\begin{flalign}
conf(\bar{x})	&= \SI{0,0407}{\percent}\pm \frac{3,182}{\sqrt{4}}*\SI{0,0030}{\percent}\\
&= \underline{\SI{0,0407}{\percent}\pm \SI{0,0048}{\percent}}
\end{flalign}

\textbf{Manuelle Bestimmung des Wassergehaltes:}\\
Beispielhaft wird nachfolgend in Gleichung (\ref{gl:wassergehalt}) die manuelle Berechnung des Wassergehaltes der flüssigen Probe Isopropanol dargestellt. Dazu wurden Messwerte der Tabelle \ref{tab:MesswerteIsopropanol} für die Berechnung verwendet. Für den Titer wurde der Mittelwert aus Tabelle \ref{tab:MesswerteTiterbestimmung} genutzt.
%Aus dem Vorgängerprotokoll___Eiheiten sind totaler schwachsinn!
%\begin{flalign}\label{gl:wassergehalt}
%	Wassergehalt &= \frac{V_{EQ}*f+DriftV*t}{m_{Probe}}\\
%	&= \frac{\SI{0,00925}{\milli\liter}*1,045+\SI{5,5}{\micro\liter\per\minute}*\SI{1,18}{\minute}}{\SI{0,4037}{\gram}}\\
%\end{flalign}

\begin{flalign}\label{gl:wassergehalt}
	W [\%]&=\frac{(V_{EQ}+t*DRIFTV)*\bar{c}+DRIFT*t}{m_{Probe}}\\[2mm]
	&=\frac{(\SI{0,00925}{\milli\liter}+\SI{1,18}{\minute}*\SI{5,5}{\micro\liter\per\minute})*\SI{5,22732}{\milli\gram\per\milli\liter}
		+\SI{28,6}{\micro\gram\per\minute}*\SI{1,18}{\minute}}{\SI{0,4037}{\gram}}*100\%\\
	&=\underline{\underline{0,029\%}}
\end{flalign}