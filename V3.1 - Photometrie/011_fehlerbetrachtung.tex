\section{Fehlerbetrachtung}
\label{sec:fehler}

Die Fehler in diesem Versuch werden hauptsächlich in Verunreinigungen Vermutet. So wurden die Bechergläser vor dem Befüllen nicht noch einmal gespült und die Vollpipetten nur mit deionisiertem Wasser und nicht mit der Probe gespült. Die Messung könnte durch Staubpartikel aus der Luft in und an den Küvetten sowie im Photometer verfälscht worden sein. Es wurde für alle Messungen die gleiche Küvette genutzt. Das schließt den Individuellen Einfluss der Wandungsbeschaffenheit als mögliche Fehlerquelle aus. 

Neben Verunreinigungen kann sich auch der unterschiedliche Verlauf der Komplexierungsreaktion der Ammoniumionen und störender Metallionen in den Ergebnissen niederschlagen. Bei der Probenvorbereitung wurden nicht alle Proben gleich behandelt. Manche Bechergläser wurden öfter als andere Verschoben. Diese unabsichtlich herbeigeführte Agitation der Lösungen könnte sich auf die darin ablaufenden Reaktionen ausgewirkt haben. 

Bei der Probenvorbereitung gehen außerdem die Toleranzen der Pipetten und Maßkolben als Fehler in das Experiment ein. 

Es konnte eine Erklärung für die auffällige Abweichung der Konzentration bei der zweiten Wiederholung der Messung der Analysenprobe A1 ausgemacht werden. Die Pipettierung der Analysenprobe A1 erfolgte \underline{nach} der Pipettierung der Probe A2. Aus der Versuchsauswertung geht hervor, dass die Probe A2 die höchst-konzentrierte ist. Es wird vermutet dass Rückstände an der Vollpipette zu einer Erhöhung des besagten Wertes beitrugen.
\cite{online}