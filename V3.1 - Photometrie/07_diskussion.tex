\newpage
\section{Diskussion}
\label{sec:diskussion}
Im folgenden wird Bezug auf gemessene und berechnete Werte des Anhangs genommen.
Die Kalibrierung mit den Kalibrierlösungen K1 und K2 ist unauffällig. Im Vergleich dazu liegen für Kalibrierlösung K3 große Abweichungen zwischen den Messergebnissen vor (siehe Anhang). Auch ein nochmaliges Vermessen der dritten Analysenlösung konnte den Unterschied nicht beheben. Es handelt sich dabei also nicht um einen Messfehler. Die Fehlerquelle lässt sich in der Probenvorbereitung vermuten.

Die mittlere Konzentration von Ammoniumionen der Probe A1 beträgt \SI{0,96}{\milli\gram\per\liter}. Der Grenzwert für Trinkwasser innerhalb der Europäischen Gemeinschaft, von \SI{0,5}{\milli\gram\per\liter}, wird damit überschritten. (vgl. Abschnitt \ref{sec:GWtest}) Auch die Proben A2 und A3 überschreiten diesen Grenzwert. Bei den Proben A1 und A2 ist dies nicht weiter von Belang, da sie nur als \glqq Laborproben\grqq\, bekannt sind. Besorgniserregend ist die Überschreitung des Grenzwertes durch die Probe A3. Selbige ist als \glqq Leitungswasser\grqq beschriftet, welches für allgemein Trinkwasserqualität besitzt. Besonders im Zusammenhang mit der, durch die WHO empfohlene, Richtkonzentration für Ammonium von \SI{0,05}{\milli\gram\per\liter} erscheint das, in diesem Versuch ermittelte, Ergebnis als überhöht. Wenngleich vom Genuss der Wässer dringend abzuraten ist, können sie, mit einer Ammoniumkonzentration kleiner als \SI{10}{\milli\gram\per\liter}, direkt in den Vorfluter kommunaler Kläranlagen eingeleitet werden. 


