\section{Theorie}
\label{sec:theorie}

\subsection{Art der Titration}
Die in diesem Versuch angewendete Art der Titration ist eine Redoxtitration. Redoxtitration zeichnen sich dadurch aus, dass sie auf einer Redoxreaktion beruhen. Redoxreaktionen sind die chemische Reaktion mit Elektronenübergang bei der ein Oxidationsmittel mit einem Reduktionsmittel reagiert.\cite{redoxreaktion} Bei einer Titration handelt es sich um eine Methode zur quantitativen Analyse einer Probe. Es wird die Konzentration eines Analyten bestimmt, indem dieser mit dem zudosierten Titriermittel zur Reaktion gebracht wird.\cite{titration} Die bekannte Konzentration des Titriermittels erlaubt anschließend, unter Annahme eines vollständigen Stoffumsatzes, die Berechnung der ursprünglichen Analytkonzentration. Der Punkt, an welchem die Reaktion des Titriermittels mit dem Analyten abgeschlossen ist, heißt Äquivalenzpunkt und kann auf eine Vielzahl von Wegen erkannt werden. Im hier betrachteten Falle kommt die bivoltametrische Indikation zum Einsatz. Deren Funktion ist  im Kapitel  \ref{sec:unterschiedBivoltBiamp} in aller Kürze dargestellt.
\subsection{Konkrete chemische Reaktion}
Die hier angewendete Redoxreaktion beruht auf der Reduktion von Iod. Daher auch der Name Iodometrie. Die Zusammengefasste Reaktionsgleichung ist als Gleichung (\ref{gl:gesamtreaktion}) nachfolgend aufgeführt. Dabei steht RN stellvertretend für die organische Base Imidazol. 
\begin{equation}\label{gl:gesamtreaktion}
	\ce{I2 + SO2 + 3 RN + CH3OH + H2O -> 2RN*HI + RN*HSO4CH3}
\end{equation}
Die zugrundeliegende Redoxreaktion ist in Gleichung (\ref{gl:kernreaktion}) dargestellt. Aus ihr geht der Zusammenhang zwischen dem Vorhandensein von Wasser und der Reduktion des Iodes hervor.
\begin{equation}\label{gl:kernreaktion}
	\ce{SO3^2- +I2 +H2O -> SO4^2- + 2H^+ +2I-}
\end{equation}
\subsection{Unterschied zwischen biamperometrischer und bivoltametrischer Titration}\label{sec:unterschiedBivoltBiamp}

Die Biamperometrie beruht auf der Messung des fließenden Stromes bei konstanter Spannung. Bei der Bivoltametrie ist es genau umgekehrt. Die Spannung wird bei konstanter Stromstärke gemessen.\cite{biamperometrie}\cite{bivoltametrie}

