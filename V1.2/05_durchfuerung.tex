\section{Durchführung}
\label{sec:durchfuerung}

Der Versuch begann mit dem Einschalten der elektronischen Bürette und des Leitfähigkeitsmessgerätes. Es wurde ohne Temperaturkorrektur durch das Messgerät gearbeitet. Die Messsonde wurde mit destilliertem Wasser gespült und in das Stativ eingespannt. Die Einstellungen an der elektronischen Bürette sind beibehalten worden. An dieser Stelle wurden mittels einer Vollpipette exakt \SI{100}{\milli\liter} der Wasserprobe abgemessen und zusammen mit einem Rührfisch in ein Becherglas mit einem Volumen von \SI{125}{\milli\liter} eingefüllt. Das Becherglas wurde auf dem Magnetrührer platziert und die Elektrode herabgesenkt bis selbige ausreichend in die Lösung eingetaucht war. Es ist wichtig, dass über dem Rührfisch genügend Raum bleibt um die Sonde gut eintauchen zu können. Das Drücken der Start-Taste an der elektronischen Bürette startet die Zugabe der 0,01 molaren Silbernitratlösung. Diese wird durch die automatische Bürette alle 15 Sekunden in Portionen von \SI{0,5}{\milli\liter} zugegeben bis ein Gesamtvolumen von \SI{20}{\milli\liter} eingebracht wurde. Nach dem Einspritzen der Silbernitratlösung wurde immer 10 Sekunden gewartet, bis der Messwert vom Messgerät abgelesen wurde. Dieser Ablauf ist 3 mal wiederholt worden, wobei für jede neue Wasserprobe ein frisches Becherglas genutzt wurde. Dann erfolgte der Wechsel von Sonde und Messgerät. Die Leitfähigkeitssonde wurde gespült und wieder in destilliertes Wasser getaucht, während die Silber-Silberchlorid-Elektrode als Einstabmesskette in das Stativ eingespannt wurde. Das Leitfähigkeitsmessgerät wurde aus- und das pH-Meter zur Messung der Potentialdifferenz eingeschaltet. Für die Potentiometrische Indikation wurden je nur \SI{50}{\milli\liter} Wasserprobe abgemessen. Da die Elektrode  nicht vom rotierenden Rührfisch getroffen werden darf und die Silberelektrode doch bis über das innenliegende Diaphragma eingetauch sein muss, wurden auch entsprechend kleinere Bechergläser genutzt. \\
Aus den vorangegangenen konduktometrischen Messungen konnte geschlossen werden, das der Äquivalenzpunkt bei etwa \SI{3,5}{\milli\liter} Silbernitratlösung zu erwarten war. Die insgesamt zu titrierte Menge wurde daher an der elektronischen Bürette von den zuvor \SI{20}{\milli\liter}, auf \SI{10}{\milli\liter} herabgesetzt. Dies geht mit einer großen Zeitersparnis und Einsparung von Chemikalien einher. Wiederum wurden die Werte nach 10 Sekunden abgelesen und notiert. Auch diesmal wurde eine Dreifachbestimmung vorgenommen.

Nach Abschluss der Messungen wurden alle silberhaltigen Abfälle in einen Sammelbehälter entsorgt und die Geräte abgewaschen. Die Silberelektrode wurde, wie die Leitfähikeitssonde auch, in ein Becherglas mit destilliertem Wasser gehangen um dem Austrocknen vorzubeugen.

Eine Bestimmung des Korrekturfaktors musste aus Zeitgründen entfallen.