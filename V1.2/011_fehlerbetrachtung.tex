\section{Fehlerbetrachtung}
\label{sec:fehler}

Die Verwendung der elektronischen Bürette eliminiert ein großes Fehlerpotential gegenüber der manuellen Zugabe der Maßlösung. Auch die digitale Anzeige an den Messgeräten lässt keine Ablesefehler zu. Die größten Fehlerquellen werden daher in den Verunreinigungen der Geräte und im Abmessen der Probenvolumina vermutet. 

Sowohl der Korrekturfaktor f$_{\text{korr}}$ als auch f$_{\text{stöch}}$ waren gegeben und wurden nicht selbst bestimmt. Diese könnten vom wahren Wert abweichen und damit einen systematischen Fehler einführen, der die Richtigkeit der Ergebnisse kompromittiert. 

