\section{Theorie}
\label{sec:theorie}
\subsection*{Löslichkeitsprodukt}
\textbf{Löslichkeitsprodukt:}
\begin{flalign}
	K_L &= c\left(A^{b+}\right)^a*c\left(A^{a-}\right)^b
\end{flalign}

\textbf{Löslichkeit:}
 \begin{flalign}
 	L &= c\left(A_mB_n\right) = \frac{1}{m}*A^{n+} =\frac{1}{n}*B^{m-}
 \end{flalign}
 Allgemein gilt auch:
 \begin{flalign}
 	L &= \sqrt[m+n]{\frac{K_L}{m^m*n^n}}
 \end{flalign}

\subsection*{\textsc{Nernst}'sche Gleichung}
\begin{flalign}
	E &= E_0 + \frac{R*T}{z*F}*\ln\left({\frac{c_\text{ox}}{c_\text{red}}}\right)
\end{flalign}

\subsection*{Konduktometrie}

\subsection*{Potentiometrie}

\subsection*{Biamperometrische und bivoltametrische Indikation}