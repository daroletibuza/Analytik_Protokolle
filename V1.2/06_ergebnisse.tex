\newpage
\section{Ergebnisse und Berechnungen}
\label{sec:ergebnisse}


\begin{figure}[h!]
	\begin{center}
		\resizebox{0.8\textwidth}{!}{
			\begin{tikzpicture}[trim axis left, trim axis right]
			\begin{axis}[
			axis lines = left,
			width = 15cm,
			height = 11cm,
			xmin = 0,
			xmax = 25,
			ymin = 500,
			ymax = 700,
			%	ytick = {-4.5,-4,...,-1},
			xtick = {0,1,2,...,24},
			ylabel={Leitfähigkeit in \si{\micro \siemens \per \centi  \meter}},
			%y label style={at={(0,0.5)}},
			xlabel={Maßlösungszugabe in \si{\milli \liter}},
			legend style={at={(0.75,0.6)},anchor=west},
			%	y dir = reverse,
			]
			\addplot [color=red, mark=*, only marks] coordinates{(0,545) (0.5,543) (1,540) (1.5,537) (2,534) (2.5,532) (3,529) (3.5,526) (4,524) (4.5,521) (5,518) (5.5,516) (6,514) (6.5,511) (7,510) (7.5,512) (8,518) (8.5,526) (9,533) (9.5,540) (10,548) (10.5,555) (11,563) (11.5,570) (12,577) (12.5,585) (13,592) (13.5,599) (14,606) (14.5,613) (15,620) (15.5,626) (16,633) (16.5,640) (17,647) (17.5,653) (18,660) (18.5,666) (19,673) (19.5,679) (20,685) };
			
			\addplot [color=blue, mark=*, only marks] coordinates{	(0,545) (0.5,544) (1,541) (1.5,538) (2,535) (2.5,533) (3,530) (3.5,527) (4,525) (4.5,522) (5,520) (5.5,517) (6,515) (6.5,513) (7,511) (7.5,514) (8,520) (8.5,528) (9,535) (9.5,543) (10,551) (10.5,558) (11,565) (11.5,573) (12,580) (12.5,587) (13,595) (13.5,602) (14,609) (14.5,616) (15,623) (15.5,630) (16,636) (16.5,643) (17,650) (17.5,657) (18,663) (18.5,670) (19,676) (19.5,683) (20,689) };
			
				\addplot [color=black, mark=*, only marks] coordinates{(0,543) (0.5,541) (1,538) (1.5,535) (2,533) (2.5,530) (3,528) (3.5,525) (4,522) (4.5,520) (5,518) (5.5,515) (6,513) (6.5,511) (7,509) (7.5,512) (8,519) (8.5,526) (9,534) (9.5,541) (10,549) (10.5,556) (11,564) (11.5,571) (12,579) (12.5,586) (13,593) (13.5,600) (14,607) (14.5,614) (15,621) (15.5,628) (16,635) (16.5,642) (17,649) (17.5,655) (18,662) (18.5,669) (19,675) (19.5,682) (20,688) };
			
			\addplot +[mark=none, dashed, blue, domain=0:25] {-5.186*x+544.817};
			\addplot +[mark=none, dashed, blue, domain=0:25] {13.928*x+409.343};
			\addplot +[mark=none, dashed, red, domain=0:25] {-5.071*x+545.483};
			\addplot +[mark=none, dashed, red, domain=0:25] {14.077*x+410.221};
			\addplot +[mark=none, dashed, black, domain=0:25] {-4.957*x+542.75};
			\addplot +[mark=none, dashed, black, domain=0:25] {14.129*x+407.993};
			
			\legend{Messreihe 1, Messreihe 2, Messreihe 3, Regressionsgeraden der Messreihe 1, Regressionsgeraden der Messreihe 2, Regressionsgeraden der Messreihe 3}
			\end{axis}
			\end{tikzpicture}
		}
		\caption{Leitfähigkeiten in Abhängigkeit der Maßlösungszugabe}
		\label{dia:kondukto}
	\end{center}
\end{figure}
\FloatBarrier

\begin{figure}[h!]
	\begin{center}
		\resizebox{0.8\textwidth}{!}{
			\begin{tikzpicture}[trim axis left, trim axis right]
			\begin{axis}[
			axis lines = left,
			width = 15cm,
			height = 11cm,
			xmin = 0,
			xmax = 25,
			ymin = 500,
			ymax = 700,
			%	ytick = {-4.5,-4,...,-1},
			xtick = {0,1,2,...,24},
			ylabel={Leitfähigkeit in \si{\micro \siemens \per \centi  \meter}},
			%y label style={at={(0,0.5)}},
			xlabel={Maßlösungszugabe in \si{\milli \liter}},
			legend style={at={(0.75,0.6)},anchor=west},
			%	y dir = reverse,
			]
			\addplot [color=red, mark=*, only marks] coordinates{(0,545) (0.5,543) (1,540) (1.5,537) (2,534) (2.5,532) (3,529) (3.5,526) (4,524) (4.5,521) (5,518) (5.5,516) (6,514) (6.5,511) (7,510) (7.5,512) (8,518) (8.5,526) (9,533) (9.5,540) (10,548) (10.5,555) (11,563) (11.5,570) (12,577) (12.5,585) (13,592) (13.5,599) (14,606) (14.5,613) (15,620) (15.5,626) (16,633) (16.5,640) (17,647) (17.5,653) (18,660) (18.5,666) (19,673) (19.5,679) (20,685) };
			
			\addplot [color=blue, mark=*, only marks] coordinates{	(0,545) (0.5,544) (1,541) (1.5,538) (2,535) (2.5,533) (3,530) (3.5,527) (4,525) (4.5,522) (5,520) (5.5,517) (6,515) (6.5,513) (7,511) (7.5,514) (8,520) (8.5,528) (9,535) (9.5,543) (10,551) (10.5,558) (11,565) (11.5,573) (12,580) (12.5,587) (13,595) (13.5,602) (14,609) (14.5,616) (15,623) (15.5,630) (16,636) (16.5,643) (17,650) (17.5,657) (18,663) (18.5,670) (19,676) (19.5,683) (20,689) };
			
			\addplot [color=black, mark=*, only marks] coordinates{(0,543) (0.5,541) (1,538) (1.5,535) (2,533) (2.5,530) (3,528) (3.5,525) (4,522) (4.5,520) (5,518) (5.5,515) (6,513) (6.5,511) (7,509) (7.5,512) (8,519) (8.5,526) (9,534) (9.5,541) (10,549) (10.5,556) (11,564) (11.5,571) (12,579) (12.5,586) (13,593) (13.5,600) (14,607) (14.5,614) (15,621) (15.5,628) (16,635) (16.5,642) (17,649) (17.5,655) (18,662) (18.5,669) (19,675) (19.5,682) (20,688) };
			
			\addplot +[mark=none, dashed, blue, domain=0:25] {-5.186*x+544.817};
			\addplot +[mark=none, dashed, blue, domain=0:25] {13.928*x+409.343};
			\addplot +[mark=none, dashed, red, domain=0:25] {-5.071*x+545.483};
			\addplot +[mark=none, dashed, red, domain=0:25] {14.077*x+410.221};
			\addplot +[mark=none, dashed, black, domain=0:25] {-4.957*x+542.75};
			\addplot +[mark=none, dashed, black, domain=0:25] {14.129*x+407.993};
			
			\legend{Messreihe 1, Messreihe 2, Messreihe 3, Regressionsgeraden der Messreihe 1, Regressionsgeraden der Messreihe 2, Regressionsgeraden der Messreihe 3}
			\end{axis}
			\end{tikzpicture}
		}
		\caption{Spannungen in Abhängigkeit der Maßlösungszugabe}
		\label{dia:potentio}
	\end{center}
\end{figure}
\FloatBarrier

\textbf{Berechnung des Mittelwertes:}
\begin{flalign}
\label{Gl:Mittelwert-Beispielrechnung1}
\bar{x} &= \frac{\sum_{n=1}^{N}x_n}{N}
\end{flalign}
\begin{flalign}
\label{Gl:Mittelwert-Beispielrechnung2}
\bar{x} &= \frac{1,32\%+1,92\%}{2}\\
&= \underline{2,12\%}
\end{flalign}

\textbf{Berechnung der Standardabweichung:}
\begin{flalign}\label{Gl:Standardabweichung-Beispielrechnung}
s &= \sqrt{\frac{\sum_{n=1}^{N}(x_n-\bar{x})^2}{N-1}}
\end{flalign}
%	\begin{footnotesize}
\begin{flalign}
s &= \sqrt{\frac{(2,32\%-2,12\%)^2+(1,92\%-2,12\%)^2}{1}}\\
&= \underline{0,285\%}
\end{flalign}
%	\end{footnotesize}

\textbf{Manuelle Berechnung der relativen Standardabweichung:}\\
Die manuelle Berechnung der relativen Standardabweichung erfolgt analog der \mbox{Gleichung (\ref{gl:S_rel})}. 
%Die mit dem Taschenrechner erhaltenen Ergebnisse sind in den Tabellen \ref{tab:MesswerteTiterbestimmung} bis \ref{tab:MesswertePoliamid} in der Sektion \glqq Manuell\grqq  \, aufgeführt. 
\begin{flalign}\label{gl:S_rel}
	s_{rel}&=\frac{s}{\bar{x}}\\
	&=\frac{0,2848\%}{2,1203\%}\\
	&=0,134321=\underline{\underline{13,43\%}}
\end{flalign}

\textbf{Berechnung des Vertrauensintervalls:}\\
\begin{flalign}
conf(\bar{x}) 	&= \bar{x}\pm \frac{t}{\sqrt{N}}*s				
\end{flalign}
\begin{flalign}
conf(\bar{x})	&= \SI{0,0407}{\percent}\pm \frac{3,182}{\sqrt{4}}*\SI{0,0030}{\percent}\\
&= \underline{\SI{0,0407}{\percent}\pm \SI{0,0048}{\percent}}
\end{flalign}

\textbf{Manuelle Bestimmung des Wassergehaltes:}\\
Beispielhaft wird nachfolgend in Gleichung (\ref{gl:wassergehalt}) die manuelle Berechnung des Wassergehaltes der flüssigen Probe Isopropanol dargestellt. Dazu wurden Messwerte der Tabelle \ref{tab:MesswerteIsopropanol} für die Berechnung verwendet. Für den Titer wurde der Mittelwert aus Tabelle \ref{tab:MesswerteTiterbestimmung} genutzt.
%Aus dem Vorgängerprotokoll___Eiheiten sind totaler schwachsinn!
%\begin{flalign}\label{gl:wassergehalt}
%	Wassergehalt &= \frac{V_{EQ}*f+DriftV*t}{m_{Probe}}\\
%	&= \frac{\SI{0,00925}{\milli\liter}*1,045+\SI{5,5}{\micro\liter\per\minute}*\SI{1,18}{\minute}}{\SI{0,4037}{\gram}}\\
%\end{flalign}

\begin{flalign}\label{gl:wassergehalt}
	W [\%]&=\frac{(V_{EQ}+t*DRIFTV)*\bar{c}+DRIFT*t}{m_{Probe}}\\[2mm]
	&=\frac{(\SI{0,00925}{\milli\liter}+\SI{1,18}{\minute}*\SI{5,5}{\micro\liter\per\minute})*\SI{5,22732}{\milli\gram\per\milli\liter}
		+\SI{28,6}{\micro\gram\per\minute}*\SI{1,18}{\minute}}{\SI{0,4037}{\gram}}*100\%\\
	&=\underline{\underline{0,029\%}}
\end{flalign}