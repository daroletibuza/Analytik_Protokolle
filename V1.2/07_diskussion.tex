\newpage
\section{Diskussion}
\label{sec:diskussion}

Aus den  Daten der konduktometrischen Indikation konnten die Reaktions- und Überschussgerade (siehe Abb. \ref{dia:kondukto}) mit einem Bestimmtheitsmaß von mindestens 0,999 (vgl. Tab. \ref{tab:konduk_regress}) erhalten werden. Deren Schnittpunkt erlaubt, durch Fällung des Lotes auf die Abzisse, das Ablesen des verbrauchten Volumens an Maßlösung. In diesem Falle wurde das Äquivalenzvolumen aber berechnet (vgl. \eqref{Gl:KonduAEQV}). Es ergibt sich dafür im Mittel \SI{7,120}{\milli\liter} für \SI{100}{\milli\liter} der Wasserprobe. 

Aus den Daten der potentiometrischen Indikation wurde der Äquivalenzpunkt durch das numerische Verfahren nach Kolthoff-Hahn bestimmt (vgl. Gl. \eqref{eq:kolthoff-hahn}). Für das Probenvolumen von \SI{50}{\milli\liter} wurde ein mittleres Äquivalenzvolumen von  Die Messdaten sind in Abb. \ref{dia:potentio} graphisch aufbereitet. Es fällt auf, dass die Messpunkte im unteren Bereich etwas weiter streuen. Die Ursache dafür wird beim Messgerät vermutet, da die relative Abweichung um einen konstanten Messfehler sich bei niedrigen Potentialen deutlich stärker auswirkt. 

Die drei Messreihen liegen sowohl bei der Konduktometrie als auch bei der Potentiometrie sehr nah bei einander. Die Messungen sind daher äußerst Präzise und wiederholgenau.