\newpage
\section{Diskussion}
\label{sec:diskussion}

%Aus den  Daten der konduktometrischen Indikation konnten die Reaktions- und Überschussgerade (siehe Abb. \ref{dia:kondukto}) mit einem Bestimmtheitsmaß von mindestens 0,999 \mbox{ (vgl. Tab. \ref{tab:konduk_regress}) }erhalten werden. Deren Schnittpunkt erlaubt, durch Fällung des Lotes auf die Abzisse, das Ablesen des Äquivalenzvolumens. In diesem Falle wurde das Äquivalenzvolumen aber berechnet (vgl. Gl. \eqref{Gl:KonduAEQV}). Es ergibt sich dafür im Mittel \SI{7,120}{\milli\liter} für \SI{100}{\milli\liter} der Wasserprobe. 
%
%Aus den Daten der potentiometrischen Indikation wurde der Äquivalenzpunkt durch das numerische Verfahren nach \textsc{Kolthoff-Hahn } bestimmt (vgl. Gl. \eqref{eq:kolthoff-hahn}). Für das Probenvolumen von \SI{50}{\milli\liter} wurde ein mittleres Äquivalenzvolumen von  \SI{3,66}{\milli \liter} ermittelt. Die Messdaten sind in Abb. \ref{dia:potentio} graphisch aufbereitet. Es fällt auf, dass die Messpunkte im unteren Bereich etwas weiter streuen. Die Ursache dafür wird beim Messgerät vermutet, da die relative Abweichung um einen konstanten Messfehler sich bei niedrigen Potentialen deutlich stärker auswirkt. 
%
%Die drei Messreihen liegen sowohl bei der Konduktometrie als auch bei der Potentiometrie sehr nah beieinander. Die Messungen sind daher präzise und wiederholgenau.

Aus den  Daten der konduktometrischen Indikation konnten die Reaktions- und Überschussgerade (siehe Abb. \ref{dia:kondukto}) mit einem Bestimmtheitsmaß von mindestens 0,999 (vgl. Tab. \ref{tab:konduk_regress}) erhalten werden. Deren Schnittpunkt erlaubt, durch Fällung des Lotes auf die Abzisse, das Ablesen des verbrauchten Volumens an Maßlösung. In diesem Falle wurde das Äquivalenzvolumen aber berechnet (vgl. Gl.\eqref{Gl:KonduAEQV}). Es ergibt sich dafür im Mittel \SI{7,120}{\milli\liter} für \SI{100}{\milli\liter} der Wasserprobe. 

Aus den Daten der potentiometrischen Indikation wurde der Äquivalenzpunkt durch das numerische Verfahren nach \textsc{Kolthoff-Hahn} bestimmt (vgl. Gl. \eqref{eq:kolthoff-hahn}). Für das Probenvolumen von \SI{50}{\milli\liter} wurde ein mittleres Äquivalenzvolumen von \SI{3,656}{\milli\liter} für \SI{100}{\milli\liter} erhalten. Zur besseren Vergleichbarkeit lässt sich dieses auf ein Probenvolumen von \SI{100}{\milli\liter} umrechnen. Für \SI{100}{\milli\liter} ergibt sich ein Äquivalenzvolumen von \SI{7,312}{\milli\liter}. Die Messdaten der potentiometrischen Indikation sind in Abb. \ref{dia:potentio} graphisch aufbereitet. Es fällt auf, dass die Messpunkte im unteren Bereich etwas weiter streuen. Die Ursache dafür wird beim Messgerät vermutet, da sich die relative Abweichung um einen konstanten Messfehler, bei niedrigen Potentialen deutlich stärker auswirkt. 


Aus den Äquivalenzvolumina kann, wie in Gl. \eqref{Gl:Chloridgehalt} dargestellt, der Chloridgehalt der untersuchten Wasserprobe berechnet werden. Die Mittelwerte des Chloridgehaltes liegen bei \SI{25,243}{\milli\gram\per\liter} und \SI{25,926}{\milli\gram\per\liter}. 




Die Einhaltung des Grenzwertes für Chlorid im Trinkwasser von \SI{250}{\milli\gram\per\liter}, wird durch einen Grenzwerttest bewiesen. 








Die drei Messreihen liegen, wie in den Abbildungen \ref{dia:kondukto} und \ref{dia:potentio} zu sehen ist, sowohl bei der Konduktometrie als auch bei der Potentiometrie, sehr nah bei einander. Die Messungen sind daher äußerst präzise und wiederholgenau. Dafür sprechen auch die relativen Standardabweichungen von nur rund 0,002. Die berechneten Chloridgehalte weichen nur um 1,5\% voneinander ab. 
Damit erweisen sich sowohl die konduktometrische, als auch die potentiometrische Indikation als gut geeignet um den Äquivalenzpunkt bei argentometrischen Untersuchungen anzuzeigen.

