\newpage
\section{Ergebnisse und Berechnungen}
\label{sec:ergebnisse}

\subsection{Kalibration}
Beginnend mit dem ersten Versuchsteil wurden die charakteristischen Absorptionsmaxima $\lambda_{1,\text{max}}$ und $\lambda_{2,\text{max}}$ bestimmt (siehe Tab. \ref{tab:maxima}).

% Table generated by Excel2LaTeX from sheet 'Daten'
\begin{table}[h!]
	\renewcommand*{\arraystretch}{1.2}
	\centering
	\caption{Absorptionsmaxima  vom Permanganat-Ion $\lambda_{1,\text{max}}$ und vom Dichromat-Ion $\lambda_{2,\text{max}}$ }
	\label{tab:maxima}
		\resizebox{10.5cm}{!}{
			\begin{tabulary}{1.0\textwidth}{C|CC}
				\hline
				\textbf{Ion} & \textbf{\ce{MnO4-} $\left(\lambda_{1,\text{max}}\right)$} & \textbf{\ce{Cr2O7^{2-}}$\left(\lambda_{2,\text{max}}\right)$}\\
				\hline
				\textbf{Wellenlänge}&\SI{525}{\nano\meter}&\SI{352}{\nano\meter}\\
				\hline		
	\end{tabulary}}
\end{table}%
\FloatBarrier

Für die Bestimmung der jeweiligen Konzentrationen in den Probelösungen, mittels linearer Kalibrierung, ist das Aufstellen der Kalibriergeraden aus den Absorptionsspektren der Kalibrierlösungen notwendig. Die Messwerte lassen sich im Protokolldeckblatt finden und sind in den Abbildungen Abb. \ref{dia:mno4} und Abb. \ref{dia:cr2o7} grafisch dargestellt.


\begin{figure}[h!]
		\begin{center}
			\resizebox{0.8\textwidth}{!}{
				\begin{tikzpicture}[trim axis left, trim axis right]
				\begin{axis}[
				axis lines = left,
				width = 15cm,
				height = 11cm,
				xmin = 0,
				xmax = 0.4,
				ymin = 0,
				ymax = 1,
				%	ytick = {-4.5,-4,...,-1},
				%	xtick = {-10,-9,...,20},
				ylabel={Absorbanz},
				%y label style={at={(0,0.5)}},
				xlabel={Konzentration in \si{\milli \mol \per \liter}},
				legend style={at={(0.025,0.87)},anchor=west},
				%	y dir = reverse,
				]
				
				\addplot [color=red, mark=*, only marks] coordinates{(0.08,0.186) (0.16,0.3685) (0.32,0.728) };
				
					\addplot [color=blue, mark=*, only marks] coordinates{(0.08,0.0994) (0.16,0.1844) (0.32,0.365) };
					
					\addplot +[mark=none, dashed, red, domain=0:0.5] {2.2567*x+0.00625};
				
				\addplot +[mark=none, dashed, blue, domain=0:0.5] {1.1098*x+0.0091};
				
			
				
				\legend{$\lambda_{1}=\SI{525}{\nano\meter}$,$\lambda_{2}=\SI{352}{\nano\meter}$,Regression $\lambda_{1} \, | \, Y=\SI{2,2567}{}*X+\SI{0,0063}{}\, | \, R^2=\SI{0,9999}{}$,Regression $\lambda_{2} \, | \, Y=\SI{1,1098}{}*X+\SI{0,0091}{} \, | \, R^2=\SI{0,9999}{}$}
				\end{axis}
				\end{tikzpicture}}
			\caption{Kalibrierkurven für das Permanganat-Ion der Wellenlängen $\lambda_1=\lambda_{1,\text{max}}$ und \mbox{$\lambda_2=\lambda_{2,\text{max}}$} }
			\label{dia:mno4}
		\end{center}
	\end{figure}
	\FloatBarrier
	
	In Abbildung Abb. \ref{dia:mno4} sind die gemessenen Absorbanzen von $\lambda_{1}$ und $\lambda_{2}$ über die Konzentration an Permanganat-Ionen aufgetragen. Für die Kalibrierpunkte der Messreihen ist jeweils eine lineare Regression durchgeführt worden, welche anhand des Bestimmtheitsmaßes in beiden Fällen als sehr genau eingestuft werden kann. Aufgrund der höheren Steigung lässt sich für die Wellenlänge $\lambda_{1}$ eine höhere Empfindlichkeit auf die Konzentration an Permanganat-Ionen feststellen als für die Wellenlänge $\lambda_{2}$. Daraus ergibt sich, dass bei gleicher Konzentration an Permanganat-Ionen fast doppelt so hohe Absorbanz für die Wellenlänge $\lambda_{1}$ als für die Wellenlänge $\lambda_{2}$ erreicht wird. Die Leerwerte $a_0$ entsprechen in dieser Darstellung den Achsenabschnitten. Zwischen beiden Messreihen ist hierfür ein geringer Unterschied von $\approx 0,003$.
	
	\begin{figure}[h!]
		\begin{center}
			\resizebox{0.8\textwidth}{!}{
				\begin{tikzpicture}[trim axis left, trim axis right]
				\begin{axis}[
				axis lines = left,
				width = 15cm,
				height = 11cm,
				xmin = 0,
				xmax = 0.5,
				ymin = 0,
				ymax = 1.5,
				%	ytick = {-4.5,-4,...,-1},
				%	xtick = {-10,-9,...,20},
				ylabel={Absorbanz},
				%y label style={at={(0,0.5)}},
				xlabel={Konzentration in \si{\milli \mol \per \liter}},
				legend style={at={(0.5,0.2)},anchor=west},
				%	y dir = reverse,
				]
				
				\addplot [color=blue, mark=*, only marks] coordinates{(0.15,0.4744) (0.3,0.9332) (0.45,1.3858)};
				
				\addplot [color=red, mark=*, only marks] coordinates{(0.15,0.0023) (0.3,0.014) (0.45,0.0088) };
				
				\addplot +[mark=none, dashed, blue, domain=0:0.6] {3.038*x+0.01973};
				
				\addplot +[mark=none, dashed, red, domain=0:0.6] {0.021667*x+0.0018667};
				
				
				
				\legend{$\lambda_{2}=\SI{352}{\nano\meter}$,$\lambda_{1}=\SI{525}{\nano\meter}$,Regression $\lambda_{2} \, | \, Y=\SI{3,038}{}*X+\SI{0,0197}{}\, | \, R^2=\SI{0,9999}{}$,Regression $\lambda_{1} \, | \, Y=\SI{0,0217}{}*X+\SI{0,0019}{} \, | \, R^2=\SI{0,9999}{}$}
				\end{axis}
				\end{tikzpicture}}
			\caption{Kalibrierkurven für das Dichromat-Ion der Wellenlängen $\lambda_1=\lambda_{1,\text{max}}$ und \mbox{$\lambda_2=\lambda_{2,\text{max}}$} }
			\label{dia:cr2o7}
		\end{center}
	\end{figure}
	\FloatBarrier
	
	In Abbildung Abb. \ref{dia:cr2o7} sind die gemessenen Absorbanzen von $\lambda_{1}$ und $\lambda_{2}$ über die Konzentration an Dichromat-Ionen aufgetragen. Ebenfalls wurde für diese Darstellung eine lineare Regression durch die Kalibrierpunkte der Messreihen bestimmt. Auch an dieser Stelle kann das Bestimmtheitsmaß für die Behauptung herangezogen werden, dass die Linearität sehr genau gegeben ist. Auffällig ist, dass der Anstieg für die Wellenlänge $\lambda_{1}$ sehr gering ist im Vergleich zu $\lambda_{2}$. Somit absorbieren Dichromat-Ionen hauptsächlich Licht der Wellenlänge $\lambda_{2}= \SI{352}{\nano\meter}$. Somit hängt die Absorbanz für die Wellenlänge vo n $\lambda_{1}=\SI{525}{\nano \meter}$ nicht maßgeblich von der Konzentration an Dichromat-Ionen ab. Die Leerwerte beider Regressionsgeraden unterscheiden sich gravierend. \\ \\
	Im Vergleich der Abbildungen Abb. \ref{dia:mno4} und Abb. \ref{dia:cr2o7} zeigt sich, dass für die Wellenlänge $\lambda_{1}$ mit \SI{525}{\nano\meter} die Empfindlichkeit zwischen Absorbanz und Konzentration für Permanganat-Ionen deutlich höher ist als für Dichromat-Ionen. Für $\lambda_{2}=\SI{352}{\nano\meter}$ ist diese Aussage genau umgekehrt zu treffen. Der hohe Leerwert der Regressionsgerade für $\lambda_{2}$ in Abb. \ref{dia:cr2o7} sollte auf mögliche Fehlerquellen untersucht werden.\\ 
	\newpage
	Es folgt die Beispielrechnung für die Empfindlichkeitsparameter $a_{11}$ und $a_12$:
	
	\begin{flalign}
	\label{gl:a11}
		a_{11}(c_1)	&= \frac{Y_1(\lambda_{1})}{X_1} = \frac{A_{\ce{MnO4-}}}{c_{1,\ce{MnO4-}}}\\
								&= \frac{0,186}{\SI{0,08}{\milli \mol \per \liter}}\\
								&= \underline{\SI{2,30}{}}
	\end{flalign}
		\begin{flalign}
		\label{gl:a12}
	a_{12}(c_1)	&= \frac{Y_2(\lambda_{1})}{X_2} = \frac{A_{\ce{Cr2O7^{2-}}}}{c_{1,\ce{Cr2O7^{2-}}}}\\
	&= \frac{0,0023}{\SI{0,15}{\milli \mol \per \liter}}\\
	&= \underline{\SI{0,0153}{}}
	\end{flalign}
		\begin{flalign}
	\label{gl:amw}
	a_{21} &= \frac{\sum_{n=1}^{N}a_{11}(c_n)}{N}\\
					&= \frac{\left(1,243+1,153+1,141\right)}{3}\\
					&= \underline{1,179}
	\end{flalign}
	
	Für die folgenden Berechnungen der Validierung und der Analyse der Proben,  wurden jeweils die Mittelwerte der Parameter $a_{11}$, $a_{11}$, $a_{11}$,  und $a_{11}$ gewählt. Diese berechneten Mittelwerte finden sich auf dem Deckblatt des Protokolls. Die Berechnungen der Parameter $a_{11}$ und $a_{12}$ finden sich in den Gleichungen Gl.\eqref{gl:a11} und Gl.\eqref{gl:a12}. Die Berechnung des Mittelwertes in Gleichung Gl.\eqref{gl:amw} nachzuvollziehen.
	
	\subsection{Validierung}
	Mittels der Validierung durch entsprechende Validierlösungen, werden nun die Differenzen zwischen den vorliegenden Konzentrationen der Validierlösungen und den berechneten Konzentrationen aus den Kalibriergeraden bestimmt. Aufgrund des Einflusses der Empfindlichkeiten werden für die Permanganat-Ionen und die Dichromat-Ionen jeweils die stärker Absorbierende Wellenlänge für die Kalibriergleichung verwendet. In den Gleichungen Gl. \eqref{gl:ber_X_cr2o7} und Gl. \eqref{gl:ber_X_cr2o7_diff} ist eine Beispielrechnung für die Validierlösung 1 dargestellt:
	\begin{flalign}
		\label{gl:ber_X_cr2o7}
		Y	&= a*X+E\\
		X	&= \frac{Y-E}{a}\\
			&= \frac{Y-0,0197}{3,038}\\
			&= \frac{0,3598-0,0197}{3,038}\\
			&= \underline{\SI{0,112}{\milli \mol \per \liter}}
	\end{flalign}
	\begin{flalign}
	\label{gl:ber_X_cr2o7_diff}
	\Delta X	&= \left|X_{\text{Soll}}-X\right|\\
	&= \left|\SI{0,075}{\milli \mol \per \liter}-\SI{0,112}{\milli \mol \per \liter}\right|\\
	&= \underline{\SI{0,037}{\milli \mol \per \liter}}
	\end{flalign}
	Alle weiteren geforderten Differenzen sind dem Deckblatt des Protokolls zu entnehmen. 
	
	\textcolor{red}{TEXT für Validierungsdifferenzen}
	
	\subsection{Analyse der Proben}
	Für die statistischen Berechnungen des Mittelwertes, Standardabweichung und des Konfidenzintervalls werden Gl.\eqref{gl:amw}, Gl.\eqref{gl:stabw} und Gl.\eqref{gl:conf}.\\
	
	\textbf{Standardabweichung:}
	\begin{flalign}
	\label{gl:stabw}
		s &=\frac{\sqrt{\sum_{n=1}^{N}(X_n-\bar{X})^2}}{N-1}\\
			&= \frac{\sqrt{(0,2234-0,2234)^2+(0,2235-0,2234)^2+(0,2234-0,2234)^2}*\si{\milli \mol \per \liter}}{3-1}\\
			&= \underline{\SI{9,22e-5}{\mmol \per \liter}}
	\end{flalign}
	
	\textbf{Konfidenzintervall:}
	\begin{flalign}
		\label{gl:conf}
		\text{conf}(\bar{X}) &= \bar{X}\pm \frac{k}{\sqrt{N}}*s\\
												&= \SI{0,2234}{\milli \mol \per \liter} \pm \frac{4,3}{\sqrt{3}}*\SI{9,22e-5}{\mmol \per \liter}\\
												&= \underline{\SI{0,2234}{\milli \mol \per \liter} \pm \SI{2,29e-4}{\milli \mol \per \liter}}
	\end{flalign}
	\subsubsection*{Ohne Berücksichtigung der Querempfindlichkeit}
		Ignoriert man die Querempfindlichkeiten bei der Berechnung der Konzentrationen so ergeben sich die entsprechenden Messwerte auf dem Deckblatt des Protokolls. Sie sind ebenfalls im Abschnitt \ref{sec:anhang} in Tab. \ref{tab:stat_oQ_m} und Tab. \ref{tab:stat_oQ_c} aufgeführt.
		\textcolor{red}{weiterer TExt fehlt !}
		
	\subsubsection*{Mit Berücksichtigung der Querempfindlichkeit}
	\textcolor{red}{Text schreiben}
	\begin{flalign}
		Y_1(\lambda_1) &= a_{11}*X_1+a_{12}*X_2\\
		X_1&=\frac{Y_1(\lambda_1) -a_{12}*X_2}{a_{11}}
	\end{flalign}
	Einsetzen von $X_1$ in $Y_2(\lambda_2)$:
	\begin{flalign}
		Y_2(\lambda_2) &= a_{21} *X_1+a_{22}*X_2\\
											&= a_{21}*\frac{Y_1(\lambda_1) -a_{12}*X_2}{a_{11}}+a_{22}*X_2\\
		Y_2(\lambda_{2})-\frac{a_{21}*Y_1(\lambda_1)}{a_{11}}&= X_2*\left(\frac{-a_{21}*a_{12}}{a_{11}}+a_{22}\right)\\
		X_2		&= \frac{Y_2(\lambda_2)-\frac{a_{21}*Y_1(\lambda_1)}{a_11}}{-\frac{a_{21}*a_{12}}{a_{11}}+a_{22}}
	\end{flalign}
	\textbf{Beispielrechnung für Probe A1:}
	\begin{flalign}
		X_2 &=  \frac{0,6023-\frac{1,1785*0,5103}{2,3010}}{-\frac{1,1785*0,0272}{2,3010}+3,1176}\\
		&=\underline{\SI{0,1101}{\milli \mol \per \liter}}
	\end{flalign}
	\begin{flalign}
		X_1 &= \frac{Y_1(\lambda_1)-a_{12}*X_2}{a_11}\\
				&= \frac{0,5103-0,0272}{2,3010}\\
				&= \underline{\SI{0,2205}{\milli \mol \per \liter}}
	\end{flalign}
	