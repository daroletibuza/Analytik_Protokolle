\newpage
\section{Ergebnisse und Berechnungen}
\label{sec:ergebnisse}

\subsection*{Kalibration}
Beginnend mit dem ersten Versuchsteil wurden die charakteristischen Absorptionsmaxima $\lambda_{1,\text{max}}$ und $\lambda_{2,\text{max}}$ bestimmt (siehe Tab. \ref{tab:maxima}).

% Table generated by Excel2LaTeX from sheet 'Daten'
\begin{table}[h!]
	\renewcommand*{\arraystretch}{1.2}
	\centering
	\caption{Absorptionsmaxima  vom Permanganat-Ion $\lambda_{1,\text{max}}$ und vom Dichromat-Ion $\lambda_{2,\text{max}}$ }
	\label{tab:maxima}
		\resizebox{10.5cm}{!}{
			\begin{tabulary}{1.0\textwidth}{C|CC}
				\hline
				\textbf{Ion} & \textbf{\ce{MnO4-} $\left(\lambda_{1,\text{max}}\right)$} & \textbf{\ce{Cr2O7^{2-}}$\left(\lambda_{2,\text{max}}\right)$}\\
				\hline
				\textbf{Wellenlänge}&\SI{525}{\nano\meter}&\SI{352}{\nano\meter}\\
				\hline		
	\end{tabulary}}
\end{table}%
\FloatBarrier

Für die Bestimmung der jeweiligen Konzentrationen in den Probelösungen, mittels linearer Kalibrierung, ist das Aufstellen der Kalibriergeraden aus den Absorptionsspektren der Kalibrierlösungen notwendig. Die Messwerte lassen sich im Protokolldeckblatt finden und sind in den Abbildungen Abb. \ref{dia:mno4} und Abb. \ref{dia:cr2o7} grafisch dargestellt.


\begin{figure}[h!]
		\begin{center}
			\resizebox{0.8\textwidth}{!}{
				\begin{tikzpicture}[trim axis left, trim axis right]
				\begin{axis}[
				axis lines = left,
				width = 15cm,
				height = 11cm,
				xmin = 0,
				xmax = 0.4,
				ymin = 0,
				ymax = 1,
				%	ytick = {-4.5,-4,...,-1},
				%	xtick = {-10,-9,...,20},
				ylabel={Absorbanz},
				%y label style={at={(0,0.5)}},
				xlabel={Konzentration in \si{\milli \mol \per \liter}},
				legend style={at={(0.025,0.87)},anchor=west},
				%	y dir = reverse,
				]
				
				\addplot [color=red, mark=*, only marks] coordinates{(0.08,0.186) (0.16,0.3685) (0.32,0.728) };
				
					\addplot [color=blue, mark=*, only marks] coordinates{(0.08,0.0994) (0.16,0.1844) (0.32,0.365) };
					
					\addplot +[mark=none, dashed, red, domain=0:0.5] {2.2567*x+0.00625};
				
				\addplot +[mark=none, dashed, blue, domain=0:0.5] {1.1098*x+0.0091};
				
			
				
				\legend{$\lambda_{1}=\SI{525}{\nano\meter}$,$\lambda_{2}=\SI{352}{\nano\meter}$,Regression $\lambda_{1} \, | \, A(X)=\SI{2,2567}{}*X+\SI{0,0063}{}\, | \, R^2=\SI{0,9999}{}$,Regression $\lambda_{2} \, | \, A(X)=\SI{1,1098}{}*X+\SI{0,0091}{} \, | \, R^2=\SI{0,9999}{}$}
				\end{axis}
				\end{tikzpicture}}
			\caption{Kalibrierkurven für das Permanganat-Ion der Wellenlängen $\lambda_1=\lambda_{1,\text{max}}$ und \mbox{$\lambda_2=\lambda_{2,\text{max}}$} }
			\label{dia:mno4}
		\end{center}
	\end{figure}
	\FloatBarrier
	
	\begin{figure}[h!]
		\begin{center}
			\resizebox{0.8\textwidth}{!}{
				\begin{tikzpicture}[trim axis left, trim axis right]
				\begin{axis}[
				axis lines = left,
				width = 15cm,
				height = 11cm,
				xmin = 0,
				xmax = 0.5,
				ymin = 0,
				ymax = 1.5,
				%	ytick = {-4.5,-4,...,-1},
				%	xtick = {-10,-9,...,20},
				ylabel={Absorbanz},
				%y label style={at={(0,0.5)}},
				xlabel={Konzentration in \si{\milli \mol \per \liter}},
				legend style={at={(0.5,0.2)},anchor=west},
				%	y dir = reverse,
				]
				
				\addplot [color=red, mark=*, only marks] coordinates{(0.15,0.4744) (0.3,0.9332) (0.45,1.3858)};
				
				\addplot [color=blue, mark=*, only marks] coordinates{(0.15,0.0023) (0.3,0.014) (0.45,0.0088) };
				
				\addplot +[mark=none, dashed, red, domain=0:0.6] {3.038*x+0.01973};
				
				\addplot +[mark=none, dashed, blue, domain=0:0.6] {0.021667*x+0.0018667};
				
				
				
				\legend{$\lambda_{1}=\SI{525}{\nano\meter}$,$\lambda_{2}=\SI{352}{\nano\meter}$,Regression $\lambda_{1} \, | \, A(X)=\SI{3,038}{}*X+\SI{0,0197}{}\, | \, R^2=\SI{0,9999}{}$,Regression $\lambda_{2} \, | \, A(X)=\SI{0,0217}{}*X+\SI{0,0019}{} \, | \, R^2=\SI{0,9999}{}$}
				\end{axis}
				\end{tikzpicture}}
			\caption{Kalibrierkurven für das Dichromat-Ion der Wellenlängen $\lambda_1=\lambda_{1,\text{max}}$ und \mbox{$\lambda_2=\lambda_{2,\text{max}}$} }
			\label{dia:cr2o7}
		\end{center}
	\end{figure}
	\FloatBarrier
	
	\textcolor{red}{TEXT}
	
	\begin{flalign}
	\label{gl:a11}
		a_{11}(c_1)	&= \frac{Y_1(\lambda_{1})}{X_1} = \frac{A_{\ce{MnO4-}}}{c_{1,\ce{MnO4-}}}\\
								&= \frac{0,186}{\SI{0,08}{\milli \mol \per \liter}}\\
								&= \underline{\SI{2,30}{}}
	\end{flalign}
		\begin{flalign}
		\label{gl:a12}
	a_{12}(c_1)	&= \frac{Y_2(\lambda_{1})}{X_2} = \frac{A_{\ce{Cr2O7^{2-}}}}{c_{1,\ce{Cr2O7^{2-}}}}\\
	&= \frac{0,0023}{\SI{0,15}{\milli \mol \per \liter}}\\
	&= \underline{\SI{0,0153}{}}
	\end{flalign}
		\begin{flalign}
	\label{gl:amw}
	a_{21} &= \frac{\sum_{n=1}^{N}a_{11}(c_n)}{N}\\
					&= \frac{\left(1,243+1,153+1,141\right)}{3}\\
					&= \underline{1,179}
	\end{flalign}
	
	Für die folgenden Berechnungen der Validierung und der Analyse der Proben,  wurden jeweils die Mittelwerte der Parameter $a_{11}$, $a_{11}$, $a_{11}$,  und $a_{11}$ gewählt. Diese berechneten Mittelwerte finden sich auf dem Deckblatt des Protokolls. Die Berechnungen der Parameter $a_{11}$ und $a_{12}$ finden sich in den Gleichungen Gl.\eqref{gl:a11} und Gl.\eqref{gl:a12}. Die Berechnung der Mittelwertes in Gleichung Gl.\eqref{gl:amw} nachzuvollziehen.
	
	\subsection*{Validierung}