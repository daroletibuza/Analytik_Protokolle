\section{Theorie}
\label{sec:theorie}

\subsection*{\textsc{Lambert-Beer}'sches Gesetz}
Die Grundlage für die photometrische Bestimmung von Konzentrationen stellt das \textsc{Lambert-Beer}'sche Gesetz dar. Dieses beschreibt den proportionalen Zusammenhang zwischen der Absorbanz $A$ und den Variablen in Form von der Küvettenschichtdicke $d$ und der Stoffmengenkonzentration $c$. Die Proportionalitätskonstante wird in diesem Fall mit dem Absorptionskoeffizienten $\alpha$ beschrieben. 
\begin{flalign}
\label{gl:abs}
A	&= \alpha*c*d
\end{flalign}
Die Absorbanz definiert sich hierbei ebenfalls durch den negativen dekadischen Logarithmus des Transmissionsgrades $T$ bzw. dem Verhältnis der gemessenen Intensität $I$ zur Ausgangsintensität $I_0$. Die Lichtquelle sendet hierbei monochromatisches Licht aus.\\
Beschrieben werden diese Zusammenhänge in den Gleichungen Gl.\eqref{gl:abs} und Gl.\eqref{gl:trans}.
\begin{flalign}
\label{gl:trans}
T	&= \frac{I}{I_0} =10^{-A}
\end{flalign}

\subsection*{Lineare Kalibrierung}
Um eine lineare Kalibrierung mit dem \textsc{Lambert-Beer}'schen Gesetz aufstellen zu können, ist es nötig die Grundgleichung (Gl.\eqref{gl:abs}) dementsprechend anzupassen. Hierfür werden weitere Größen eingeführt bzw. das \textsc{Lambert-Beer}'sche Gesetz umgeformt. Die Signalgröße $Y$ entspricht dabei dem Verhältnis zwischen Absorbanz $A$ und der Schichtdicke der Küvette $d$. Die Proportionalitätskonstante entspricht nun der Empfindlichkeit der Methode $a$.
\begin{flalign}
\tag*{| : d}
	A&= \alpha*c*d\\
	Y&=\frac{A}{d}= a *c
\end{flalign}
Zusätzlich  wird die Fehlergröße $E$ additiv hinzugefügt.
\begin{flalign}
	Y&=a *c+E
\end{flalign}
Diese Fehlergröße $E$ unterliegt der Annahme, dass durch eine weitere Komponente mit einer Konzentration $X_2$ ebenfalls monochromatisches Licht der Wellenlänge $\lambda_1$ der Komponente 1 absorbiert. Für die Signalgröße $Y{\lambda_1}$ ergibt sich daraus:
\begin{flalign}
	Y(\lambda_1)	&= a_1(\lambda_1)*X_1+a_2(\lambda_2)*X_2
\end{flalign}

Um die nötigen Empfindlichkeiten $a_1$ und $a_2$ zu bestimmen, sind weitere Messschritte erforderlich. So müssen die Wellenlängen $\lambda_{1,\text{max}}$ und $\lambda_{2,\text{max}}$ bestimmt werden, bei welchen die einzelnen Komponenten die maximale Absorbanz aufzeigen. Dies erfolgt mittels Spektroskopie. Infolgedessen lässt ein Gleichungssystem für bis zu zwei Komponenten oder eine Matrix für mehr als zwei Komponenten aufstellen, welche es mathematisch zu lösen gilt. Für die Wellenlängen $\lambda_1$ und $\lambda_2$ ergibt sich daraus ein lineares Gleichungssystem der Form:
\begin{flalign}
\label{gl:lgs1}
	Y_1(\lambda_1) &= a_{11}*X_1+a_{12}*X_2
\end{flalign}
\begin{flalign}
\label{gl:lgs2}
Y_2(\lambda_2) &= a_{21}*X_1+a_{22}*X_2
\end{flalign}

Zur Vereinfachung der Kalibration werden die Empfindlichkeiten für den Einkomponentenfall mit den Empfindlichkeiten im Zweikomponentenfall gleichgesetzt.
Im weiteren Vorgehen werden die Kalibriergeraden mit Hilfe von Validierlösungen überprüft. Diese beinhalten bekannte Konzentrationen an Kaliumpermanganat und Kaliumdichromat. Nach erfolgreicher Validierung der Kalibrierkurven, kann mit der photometrischen Bestimmung der Probenkonzentrationen $X_1$ und $X_2$ begonnen werden. Für die Umrechnung der jeweiligen Absorbanz in  die Konzentration ist die Lösung des Gleichungssystems aus Gl. \eqref{gl:lgs1} und Gl.\eqref{gl:lgs2} nötig.