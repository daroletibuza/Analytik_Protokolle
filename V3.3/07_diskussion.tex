\section{Diskussion}
\label{sec:diskussion}

Es stellt sich heraus, dass die Einbeziehung von Querempfindlichkeiten in die Auswertung der Messungen ein sinnvoller Schritt ist. Durch Querempfindlichkeiten können, je nach Matrix der Lösung, entscheidende Fehler in der Bewertung von Messungen erzeugt werden.\\
Für die gemessene Konzentration an Permanganat-Ionen ist dieser Einfluss etwas geringer als für die Dichromat-Ionen. In beiden Fällen sollte jedoch eine Korrektur mittels Einbeziehung der Querempfindlichkeit durchgeführt werden. Für die Permanganat-Ionen-Konzentration ergeben sich sonst Fehler bis zu \SI{57}{\percent} vom Sollwert, statt \SI{10}{\percent}. Gravierender sind diese Unterschiede in der Dichromat-Ionen-Konzentration. auch mit Korrektur empfiehlt es sich eine weitere Messung der Proben durchzuführen.\\
Da in diesem Versuch zwei Komponenten mit gegenseitigen Querempfindlichkeiten getestet wurden, ist es nötig aufgrund der beschriebenen Abweichungen in der Fehlerrate, diese Querempfindlichkeiten mit einzuberechnen. So ergibt es sich, dass für die Methode der simultanen Mehrkomponentenphotometrie  Kalibriergeraden der untereinander querempfindlichen Komponenten aufzunehmen und rechnerisch einzubeziehen. \\
In diesem Versuch zeigte sich die Mehrkomponentenphotometrie als eine präzise Messmethode, jedoch mit starker Anfälligkeit für eine ungenaue Arbeitsweise unter der die Richtigkeit der Messwerte leiden kann. Somit eignet sich diese Methode für multiple Bestimmung von Komponenten einer Lösung, erfordert jedoch einige Erfahrung in der Versuchsdurchführung.\linebreak
Um die Betrachtung von Querempfindlichkeiten zu umgehen, wäre es möglich die bekannten Komponenten zuvor voneinander zu trennen um diese separat analysieren zu können. Dieser Schritt kann, neben dem erhöhten Arbeitsaufwand, auch ungewollte Fehlerquellen und Ungenauigkeiten zur Folge haben.