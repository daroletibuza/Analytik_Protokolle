\newpage
\section{Durchführung}
\label{sec:durchfuerung}
Im ersten Teil der Versuchsdurchführung werden aus den Standardlösungen für Kaliumdichromat $\left(c_\text{\ce{KMnO4}}=\SI{0,02}{\mol \per \liter}\right)$ und Kaliumpermanganat  $\left(c_\text{\ce{K2Cr2O7}}=\SI{0,0015}{\mol \per \liter}\right)$ jeweils drei Kalibrierlösungen hergestellt. Es werden jeweils \SI{100}{\milli \liter} Kalibrierlösung, in verschiedenen Maßkolben, hergestellt. Zusätzlich werden, für die Überprüfung der Kalibrierkurven, drei verschiedene Validierlösungen angemischt. Sowohl die Konzentrationen der Validierlösungen, als auch die der Kalibrierlösungen, finden sich im Protokolldeckblatt oder in der Praktikumsanleitung zu diesem Versuch wieder.\\
Im zweiten Versuchsteil wird mit der Aufnahme von Spektren begonnen. Nach der Bestimmung der Basislinie mit einer Küvette, welche lediglich destilliertes Wasser enthält, wird diese vom Messsystem automatisch gespeichert. Von allen folgenden Messungen wird das Spektrum der Basisllinie subtrahiert, um den Einfluss der Matrix herauszurechnen. \linebreak
Um die charakteristischen Wellenlängen des Permanganat-Ions $(\lambda_{1,\text{max}})$ und des Dichromat-Ions $(\lambda_{2,\text{max}})$ zu bestimmen, werden jeweils die Kalibrierlösungen mittlerer Konzentration (K2) im Spektrometer untersucht und am PC ausgewertet. Für die folgende photometrische Bestimmung wird sich auf diese bestimmten Wellenlängen bezogen.\\
Im dritten Versuchsteil werden nun mit Hilfe der bestimmten Absorptionsspektren $(\lambda_{1,\text{max}})$ und $(\lambda_{2,\text{max}})$ die Kalibriergeraden aus den sechs Kalibrierlösungen bestimmt. Es wird hierbei eine Dreifachbestimmung durchgeführt. Die Überprüfung der Kalibriergeraden erfolgt im Anschluss mit den zuvor hergestellten Validierlösungen.\\
Im letzten Versuchsteil werden die Kalibrierkurven genutzt, um aus den Messergebnissen der Probe 1 und 2 die Konzentrationen an Permanganat- und Dichromat-Ionen zu erhalten. Auch an dieser Stelle werden die Messwerte für die Absorbanz in Dreifachbestimmung ausgeführt.